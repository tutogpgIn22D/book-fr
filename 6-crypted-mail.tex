\chapter{Lire et écrire du courrier chiffré}

Bon, j'ai dû vous envoyer un courriel chiffré. Donc lisible seulement
par vous.

\section{Ce que j'ai fais}\label{ce-que-jai-fais}

Lorsque j'ai récupéré votre clé publique, je l'ai mise dans mon
trousseau de clés.\\Ceci m'a permis de l'analyser, et donc de vous
indiquer que, oui, \emph{votre clé est bien générée}.

\section{Conséquences}\label{consuxe9quences}

Puisque j'ai votre clé publique, je peux désormais vous envoyer des
messages chiffrés.

\textbf{Avec un courriel chiffré, vous avez la certitude que vous seul
avez pu lire ce message. En revanche vous ne pouvez être sûr de
l'expéditeur.}

Par contre, si je signe également mes messages, vous ne pouvez pas
vérifier ma signature.\\En effet, encore une fois, pour vérifier ma
signature, vous avez besoin de ma clé publique, et je ne vous ai pas
encore indiqué comment la récupérer.

\textbf{La signature est une preuve d'authenticité du message, donc de
l'expéditeur.}

C'est pour cette raison que votre logiciel de courriel vous indique
sûrement que le courriel que je vous ai envoyé est signé par gpg, mais
qu'il ne peut en vérifier la signature.

\section{Un petit accroc}\label{un-petit-accroc}

Attention ! Seul le corps du message est chiffré ! Le sujet, l'en-tête
du message, ainsi que les méta-données (d'où le message a été envoyé, à
qui, et par où il est passé) sont en clair et lisibles de tous. Il est
difficile de faire autrement: comment les serveurs de messagerie
pourraient-ils savoir à qui remettre le message si la destination n'est
pas lisible ?

Il est donc recommandé de mettre un sujet assez neutre et générique si
vous voulez assurer une bonne confidentialité.

Quelque chose comme «Un bon plan» plutôt que «le plan de domination du
monde», ou «les derniers chiffres de production» plutôt que «chiffres de
prod' en hausse de 50 \%: on a tout bon !»

\section{Comment récupérer la clé publique ?}\label{comment-ruxe9cupuxe9rer-la-cluxe9-publique}

\subsection{Par courriel}\label{par-courriel}

J'aurais pu vous envoyer la clé publique du tutoriel, de la même façon
que vous m'avez envoyé la votre.

Mais ce n'est en fait pas très sûr: qu'est-ce qui garanti qu'une
personne malveillante n'a pas détourné votre courriel et remplacé votre
clé par une autre ?

C'est le genre de réflexions que je souhaite vous voir développer. La
sécurité sur internet est un processus, une manière de penser.

\subsection{Sur une page web}\label{sur-une-page-web}

Pour les personnes qui ont une page web, vous pouvez mettre votre clé à
disposition dessus. Il est de bon ton d'indiquer aussi l'empreinte de la
clé dans la page en question.

Une empreinte de clé gpg est une suite de caractères alphanumériques
propre à chaque clé. Une empreinte permet donc à la fois d'identifier de
façon (relativement) sûre une clé et de garantir qu'elle est bien
intègre, que personne ne l'a corrompue.

Il est intéressant de voir qu'on utilise un vocabulaire réservé
d'habitude aux humains : \emph{intégrité} et \emph{corruption}. Il
s'agit bien d'indiquer des notions de confiance, d'absence de doute, de
probité.

Votre gestionnaire de clé vous donnera cette empreinte qui ressemblera à
ceci :

\begin{verbatim}
30CF 1DA5 7E87 6BAA 730D E561 42E0 A02E F1C9 35A4
\end{verbatim}

Cette empreinte est celle de la clé publique de \emph{Tuto-gpg @
22decembre.eu}.\\C'est le même principe que les sommes de contrôle MD5\footnote{D'ailleurs les sommes de contrôle MD5 sont encore utilisées par OpenPGP, sauf que l'algorithme MD5 est aujourd'hui considéré comme obsolète. Le sujet sera évoqué dans la suite du tutoriel. Vous pouvez en apprendre davantage sur \url{http://fr.wikipedia.org/wiki/MD5}.} des
fichiers téléchargés sur internet - typiquement une iso de distribution Linux.

Certaines personnes mettent aussi l'empreinte de leur clé gpg sur leur
carte de visite, pour pouvoir les distribuer plus facilement.

\subsection{Les serveurs de clés}\label{les-serveurs-de-cluxe9s}

En connaissant une empreinte de clé gpg, on peut demander à son
gestionnaire de clés de trouver la clé sur internet ! En fait sur des
serveurs que l'on appelle «serveurs de clés» ou «serveurs gpg».

Voici quelques adresses de serveurs :

\begin{itemize}
\itemsep1pt\parskip0pt\parsep0pt
\item
  hkp://keyserver.ubuntu.com/
\item
  hkp://pool.sks-servers.net/ \footnote{Ce serveur est en fait un pool de serveurs de clés distribué par round-robind DNS. C'est le \textit{serveur} de clé le plus utilisé aujourd'hui.}.
\item
  hkp://pgp.mit.edu/
\item
  hkp(s)://keys.gnupg.net/
\end{itemize}

Vous pouvez indiquer à gpg d'utiliser en priorité le serveur que vous
préférez grâce à la configuration de votre gestionnaire de clés.

Le S indique que vous pouvez utiliser ce serveur avec une connexion
sécurisée par TLS. L'avantage, si vous êtes parano, c'est que personne
ne sait quelles clés vous recherchez.

L'intérêt de ces serveurs c'est de vous permettre de publier votre clé,
et de recevoir des clés et des messages signés et/ou chiffrés de la part
de personnes que vous ne connaissez pas.

\section{Exercice}\label{exercice}

En guise d'exercice aujourd'hui, je vais vous demander de récupérer la
clé publique du tutoriel, et de m'envoyer un courriel chiffré.

\subsection{Récupérer la clé
publique}\label{ruxe9cupuxe9rer-la-cluxe9-publique}

Vous l'avez compris, il s'agit de vous faire comprendre le
fonctionnement des serveurs de clés.

Pour se faire, vous devez ouvrir la boite de dialogue avec le serveur de
clé de votre gestionnaire de clés. Le logiciel vous proposera de faire
une recherche avec une chaîne de caractères. Autrement dit, une
empreinte, ou une adresse courriel.

Soit vous copiez-collez l'empreinte de la clé du tutoriel (indiquée
juste au dessus) ou l'adresse courriel dans la boite de dialogue.

Votre gestionnaire de clés va alors vous proposer une ou plusieurs clés
à télécharger.

Si vous avez indiqué l'adresse courriel, vérifiez que l'empreinte est
bien la bonne.\\Inversement, si vous avez indiqué l'empreinte, vérifiez
qu'il s'agit bien de la bonne adresse !\\En effet, grâce à ces
empreintes, on peut vérifier que la clé que l'on a téléchargée est bien
celle qu'on cherchait.

Justement, j'ai créée plusieurs jeux de clés, non pas pour vous piéger,
mais pour vous faire réfléchir et travailler (travailler votre logique,
je ne retire aucun revenu de ce tutoriel - mais si vous trouvez qu'il
est utile, vous pouvez me flattrer - c'est dans la colonne de gauche).

Le but de ce tutoriel c'est de vous apprendre à utiliser GPG, donc il
vous faut l'utiliser pour le comprendre.

Au passage, vous pouvez noter que ces personnes ont signé cette clé:

\begin{itemize}
\itemsep1pt\parskip0pt\parsep0pt
\item
  \href{http://alterlibriste.free.fr/}{alterlibriste}
\item
  \href{http://lehollandaisvolant.net/}{le hollandais volant}
\end{itemize}

Je me suis en effet inspiré de leurs textes, ou ils m'ont encouragé à
écrire ce tutoriel. Je profite donc de ce moment pour les remercier,
ainsi que:

\begin{itemize}
\itemsep1pt\parskip0pt\parsep0pt
\item
  \href{http://genma.free.fr/}{genma}
\item
  \href{https://maymay.net/}{Maymay} qui m'a entre autre aidé avec
  quelques uns des articles anglais.
\end{itemize}

Regarder maintenant le courriel que je vous ai envoyé: votre logiciel
vous indique sûrement que la signature est valide, non ?

\subsection{Envoyer un courriel
chiffré}\label{envoyer-un-courriel-chiffruxe9}

Il faut donc que vous m'écriviez un courriel chiffré. Vous pouvez aussi
le signer, mais cela n'a pas d'incidence.

Juste avant de l'envoyer donc, cliquez sur le bouton \emph{Chiffrer} ou
selectionner l'option qui va bien. Voila !

Il est à noter que comme vous avez récupéré la clé publique du tutoriel,
il est très probable que votre logiciel de courriel vous ait proposé de
chiffrer le message lors de sa rédaction !
