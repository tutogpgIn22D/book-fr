\chapter{Signer des clés}

Je souhaite que vous puissiez commencer à utiliser activement gpg, au
moins avec des amis proches ou des membres de votre famille qui auraient
lu ce tutoriel, et pour cela, il faut que vous puissiez signer des clés.

\section{Signer des clés\ldots{}}\label{signer-des-cluxe9s}

Oui, avec gpg, on peut signer des courriels, des fichiers, mais aussi
des clés gpg !

\emph{Délicieusement récursif}

Lorsque vous signez une clé, vous accordez un certain crédit à celle-ci,
et vous l'indiquez également à tous ceux qui récupéreront votre
signature.

La signature d'une clé indique en fait que vous considérez que le
propriétaire de cette clé est bien légitime.

On va parler, là de \emph{cercle de confiance immédiat} (CCI pour
résumer): ce sont toutes les personnes que vous avez rencontré, dont
\textbf{vous} avez vérifié l'identité et signé la clé.

Retenez bien qu'il s'agit d'un concept que je définis ici, pour les
besoins de l'article, pour vous permettre de comprendre toutes les
notions ci dessous.

\section{Un problème d'identité}\label{un-probluxe8me-didentituxe9}

On va prendre un exemple: quelqu'un se présente pour faire signer sa
clé. Appelons le John.

Vous devez d'abord vérifier que John est bien le propriétaire de cette
clé, et pour se faire, vérifier avec lui l'empreinte de la clé.

Vous devez ensuite vérifier que John est bien qui il prétend être. Un
coup d'oeil (soigneux) à ses papiers d'identité vous le
confirmera\ldots{} mais pas seulement !\\Car l'identité, c'est bien plus
qu'un simple bout de carton plastifié, fut-il protégé avec des encres
spéciales\ldots{}

L'identité, c'est aussi les diverses fonctions qu'on occupe: trésorier
d'une association, blogueur plus ou moins anonyme, présence sur des
réseaux sociaux.

Tout ce qui renforce la confiance dans l'identité qu'une personne nous
indique peut être vérifié.

Une fois ces divers points validés, vous êtes à peu près sûr qu'il
s'agit bien de la bonne clé et de la bonne personne. Vous pouvez donc
signer sa clé.

\section{Confiance}\label{confiance}

En signant une clé, vous accordez également un certain niveau de
confiance dans son propriétaire.

Ce sont deux choses liées mais néanmoins distinctes, et il est important
de le comprendre.

Il y a cinq niveaux de confiance différents:

\begin{itemize}
\itemsep1pt\parskip0pt\parsep0pt
\item
  indéterminée/je ne sais pas (c'est le défaut)
\item
  absence de confiance/nulle
\item
  partielle
\item
  totale
\item
  absolue/c'est ma clé
\end{itemize}

Ce niveau de confiance traduit quelle confiance vous avez dans ce
propriétaire pour, lui-même, vérifier les identités et donc assurer son
cercle de confiance immédiat (CCI). Et ultimement, quelle confiance vous
accordez à son jugement à lui sur ceux d'autres personnes !

Par exemple, vous pouvez avoir vérifié soigneusement la clé d'un membre
de votre famille, un cousin mettons. Vous avez donc absolument confiance
dans \textbf{\emph{cette}} clé !

Mais par contre, vous pensez que votre cousin n'est pas encore bien rodé
avec OpenPGP. Peut-être qu'il signe un peu n'importe qui, ou qu'il
accorde une confiance excessive.

Vous n'avez donc aucune confiance dans son CCI.

Donc vous signez sa clé, mais avec une confiance nulle.

Et cette confiance nulle est quelque chose que vous pouvez (\emph{devez}
!) indiquer sans honte ni crainte. Pour ce faire, le niveau de confiance
est codé, inscrit dans la signature, de telle façon qu'il soit et reste
une information \emph{privée}.

\textbf{\emph{Vous êtes libre de votre jugement sur les gens dont vous
signez la clé ! Il s'agit là d'un cas de liberté d'expression et
d'opinion.}}

Plus vous respecterez ces diverses considérations, plus votre propre
jugement sera respecté par d'autres, en particulier des gens que vous
connaissez bien.

Et donc plus votre CCI prendra, lui une valeur haute également.

Pour ma part, je pense qu'on devrait signer avec une confiance partielle
par défaut, et réserver la confiance totale aux gens dont on connaît le
sérieux.

Votre CCI est donc à la fois constitué des clés (et donc de leurs
propriétaires) que vous avez signées, mais également du niveau de
confiance qui leur est accordé !

\section{Des exemples}\label{des-exemples}

Vous rencontrez quelques personnes dans un bar lors d'un rendez-vous de
votre LUG.

\subsection{Arthur}\label{arthur}

Arthur vous explique qu'il débute dans l'utilisation de GPG et souhaite
donc se construire une toile de confiance. \emph{Ok}.\\Vous vérifiez
correctement son identité et il fait de même avec vous. Puis, chez vous,
vous signez sa clé, puisque vous l'avez vérifiée.

En revanche vous lui accordez une confiance nulle, car vous estimez que
sa pratique de Gpg est encore trop récente.

Cette confiance nulle ne vous empêche pas, un mois plus tard, lors d'une
autre rencontre, de lui demander comment il gère son trousseau de clé,
de constater son sérieux et d'élever son niveau de confiance.

Cette confiance nulle n'empêche pas non plus cette personne de vous
accorder un jugement haut ou bas également.\\Et les signatures d'autres
personnes sur sa clé s'ajoute à la votre. Signatures peut-être plus
positives !

\emph{Il ne faut pas que vous vous sentiez honteux ou irrespectueux
d'indiquer une confiance basse à Arthur.}

\subsection{Miriam}\label{miriam}

Miriam vous explique qu'elle est développeuse Debian. Vous vérifiez son
identité et sa clé.\\Chez vous, vous vérifiez également ses dires au
sujet de Debian et vous constatez qu'elle a bien dit la vérité.

Donc vous signez sa clé, et vous lui accordez une confiance totale car
vous savez que les dev' Debian utilisent beaucoup GPG et doivent
respecter un certain sérieux.

\subsection{Greg}\label{greg}

Greg vous indique qu'il utilise très régulièrement Gpg. Après avoir
vérifié sa clé, vous la signez, avec une confiance
partielle.\\Simplement parce que vous ne le connaissez pas, mais vous
voyez Greg comme quelqu'un de sérieux. Sans plus.

\subsection{Karolina}\label{karolina}

Karolina vous explique qu'elle a adopté un système de signature très
précis décrit avec beaucoup de détails dans un document disponible sur
son blog (on parle là de \emph{politique de signatures}).

Vous trouvez sa démarche très juste et décidez de lui accorder une
confiance totale, précisément à cause de ce système de signatures dont
vous estimez le fonctionnement très équilibré.

\section{La Toile de confiance}\label{la-toile-de-confiance}

C'est quoi la toile de confiance (les anglophones parlent de \emph{Web
of Trust}) ?

La toile de confiance, ce sont tous les CCI additionnés et mis bout à
bout, constituant des chaînes:

Vous avez mis une confiance partielle en Greg. Donc son CCI se retrouve
lui aussi avec une confiance partielle.

Vous avez mis une confiance complète en Miriam. C'est comme si vous
aviez fait entrer tous ses contacts dans votre cercle de confiance à
vous.\\Et c'est Miriam qui a défini si oui ou non vous pouvez faire
confiance dans telle ou telle personne que vous n'avez jamais rencontré
- puisque vous avez confiance dans le jugement de Miriam.\\Si, donc,
elle a mis elle-même quelqu'un d'autre en confiance totale, alors cette
troisième se retrouve à nouveau \emph{de facto} dans votre cercle de
confiance \emph{étendu}.

Seulement, et c'est bien important, il ne \emph{faut pas que
}\textbf{vous}* signiez* une des clés signées par Miriam sans avoir
rencontré la/le propriétaire de cette clé.\\Miriam l'a fait. Donc la clé
qu'elle a signée sera reconnue comme valide par votre gestionnaire de
clés.\\D'ailleurs pourquoi signeriez vous une clé sans avoir rencontré
la personne ?

Gpg va déterminer par un algorithme jusqu'à quel point vous pouvez faire
confiance à telle clé, que vous n'avez pas signé, à travers le mécanisme
de la toile de confiance.

Par défaut, Gpg n'ira pas plus loin que cinq cercles de confiance. Et il
faut trois signatures avec une confiance partielle, ou une signature
avec une confiance complète pour qu'une clé soit valide.

\subsection{Corollaires}\label{corollaires}

Il faut se rendre compte également que la Toile de confiance doit être
considérée avec sérieux.

Par exemple, les opinions politiques, religieuses, son origine, son
sexe, sa couleur de peau ni même votre proximité sentimentale (ami,
membre de la famille\ldots{}) avec le propriétaire de la clé n'ont rien
à voir ici.\\Seul compte le sérieux qu'il accorde à son CCI.

Si vous rencontrez quelqu'un qui vous explique qu'il signe des gens
d'une manière très précise, avec une politique de signature détaillée,
vous pouvez être admiratif de son sérieux et décider de lui accorder une
confiance totale.\\Puis au cours de la discussion, vous vous rendez
compte que c'est un pédophile néonazi pronnant l'eugénisme et mangeant
des chatons au petit-déjeuner avec de la sauce soja, vous \emph{devriez
quand même} signer sa clé !

%\href{http://giphy.com/gifs/cat-black-and-white-food-S4IWCPAjhbzUc}{\includegraphics{http://i.giphy.com/S4IWCPAjhbzUc.gif}}

Bien sûr, vous le dénoncer à la police aussi sec, mais les deux actions
sont compatibles.

\subsection{Quelle est ma responsabilité dans tout ça ?}\label{quelle-est-ma-responsabilituxe9-dans-tout-uxe7a}

Il s'agit donc bien d'un système, d'un réseau de confiance
\emph{relative}.

Il n'y a pas d'autorité centrale de type étatique qui indique quelle
identité est vraie ou fausse.

C'est vous qui devez estimer quelle confiance vous accordez et à qui
vous l'accordez.

Ne soyez toutefois pas effrayé d'une telle responsabilité !

Le fonctionnement du réseau est démocratique: si vous attribuez un
mauvais degré de confiance à un utilisateur, votre ``vote'' sera
contrebalancé par ceux des autres.

Les \emph{politiques de signature} documentées et accessibles en ligne
renforcent cet aspect démocratique, parce qu'elles sont impartiales,
comme décrit au dessus (cf le néonazi mangeur de chatons).

%\href{https://www.flickr.com/photos/jmtimages/3286566742/}{\includegraphics{https://farm4.staticflickr.com/3615/3286566742_c673e4845d_z.jpg}}

La responsabilité est donc individuelle et distribuée. C'est en fait du
pair-à-pair ! C'est nous tous qui, collectivement, sommes l'autorité en
laquelle nous faisons confiance pour valider les clés.

\section{Tromperies ?}\label{tromperies}

\subsection{Que faire si quelqu'un tente de me tromper en me faisant
signer une fausse clé
?}\label{que-faire-si-quelquun-tente-de-me-tromper-en-me-faisant-signer-une-fausse-cluxe9}

Déjà, il faut bien voir qu'en signant une clé, vous validez aussi une
adresse courriel. Donc c'est aussi l'adresse courriel qu'il s'agit de
vérifier.

Et le plus simple dans ce cas est que la personne vous envoie un
courriel signé depuis cette adresse. (Il peut le faire quelques heures
plus tard, ce n'est pas un soucis).\\De cette façon, vous êtes sûr que
vous signiez la clé associée à cette adresse courriel, et qu'elle
appartient à la personne que vous avez rencontré.

C'est toutefois un poil lourd comme procédure. À vous de voir.

\subsection{Il peut arriver qu'on veuille écrire à quelqu'un qu'on ne
connaît pas ! Comment être sûr qu'on récupère la bonne clé
?}\label{il-peut-arriver-quon-veuille-uxe9crire-uxe0-quelquun-quon-ne-connauxeet-pas-comment-uxeatre-suxfbr-quon-ruxe9cupuxe8re-la-bonne-cluxe9}

C'est le rôle de la toile de confiance, décrite au dessus.

Si vous souhaitez écrire à Valentina, et que des gens mal-intentionnés
ont crée une ou des fausse(s) clé(s) qu'ils ont envoyés sur les serveurs
de clés, comment reconnaître la bonne ?

Il y a de fortes chances que la clé de Valentina soit celle qui comporte
le plus de signatures.\\Et, très important, plus ces signatures viennent
de personnes diverses (de nationalités, de fonctions diverses\ldots{}),
plus la clé sera sûre !

Il est par exemple très facile de créer des clés avec vingt signatures
dessus.

Il est par contre assez difficile de créer une clé avec une ou deux
\emph{fausses} signatures de dev' Debian. Or il est très facile
d'obtenir la signature d'un dev' Debian!\\Non pas que ce soient des gars
ou des filles \emph{faciles}. Mais ils sont nombreux et éparpillés un
peu partout.\\Si vous habitez une grande ville occidentale, il est
probable que vous ayez \emph{un dev' Debian à portée de main} et qu'il
signera votre clé (en suivant sa politique de signature) si vous
l'invitez pour une bière dans un bar.

S'il est donc facile de créer des clés avec de fausses signatures, il
est tout aussi facile de faire signer sa clé par une personne publique à
l'identité vérifiable (dev' Debian ou dev' d'autres logiciels libres,
blogueur, membre d'une association\ldots{}).

Et c'est donc cette dernière clé que vous utiliserez en cas de doute !

\section{Exercice}\label{exercice}

Bon, c'est bon, vous pigez le truc de la toile de confiance ?

Ne vous inquiétez pas, vous aurez d'autres articles qui donneront
d'autres détails. Et vous pouvez toujours revenir ici lire encore et
encore cet article.

Je vais vous demander pour l'exercice du jour, de signer la clé du
tutoriel, d'y mettre la confiance qui vous paraît correcte, puis de me
l'envoyer par courriel.

Nous sommes bien d'accord qu'il ne faut pas faire cela normalement,
puisque qu'on ne s'est pas rencontré.

C'est la raison pour laquelle j'ai indiqué dans un des articles que la
clé du tutoriel ne sera utilisée que pour les besoins du tutoriel.

\subsection{Donc, comment qu'on fait ?}\label{donc-comment-quon-fait}

Dans votre gestionnaire de clés, vous devez sélectionner la clé en
question, puis l'option \textbf{\emph{Signer}}.

Le logiciel vous demandera quelle confiance vous souhaitez accorder à
cette clé. Si vous avez plusieurs clés privées, il vous sera demandé
avec laquelle vous souhaitez signer.

Vous pouvez également passer par l'édition des propriétés de la clé et
changer simplement le niveau de confiance. La signature sera alors faîte
automatiquement.

Vous exportez ensuite la clé dans un fichier, que vous envoyez en pièce
jointe de courriel à \emph{Tuto-gpg @ 22decembre.eu}. Lorsque je vais
récuperer la clé, les signatures vont s'additionner les unes les autres.
Je vais alors vous répondre et vous envoyer votre clé, signée avec la
clé du tutoriel.

Summum de la feignantise, KGpg et d'autres logiciels vous proposent de
faire le tout s'un seul coup: avec l'option \textbf{\emph{Signer et
envoyer par courrier electronique}}, KGpg va effectivement signer
suivant vos directives, puis préparer le courriel avec la clé dedans.

\subsection{Précisions}\label{pruxe9cisions}

Certains logiciels (notamment KGpg) mettent l'accent sur la vérification
de l'identité du propriétaire de la clé pour vous indiquer le niveau de
confiance que vous pouvez accordez.

Il s'agit là en fait d'une assertion selon laquelle, puisque l'identité
et la clé ont bien été validées sérieusement, il est raisonnable de
penser que le propriétaire de la clé accorde le même sérieux dans
d'autres signatures.

Ceci ne change pas la signification de votre signature et ma description
reste valable: la confiance accordée indique à quel point vous estimez
le propriétaire de la clé capable d'entretenir son CCI.

C'est donc ce niveau de confiance que vous devez indiquer.

Ces logiciels ont par exemple une ergonomie trompeuse. Si vous avez des
questions, n'hésitez pas à lire et relire cet article ou à m'écrire.