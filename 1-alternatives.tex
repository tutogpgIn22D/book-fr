\chapter{Existe t-il des alternatives sérieuses à l'utilisation de GPG ?}

\emph{NB : Cet article est plutôt destiné aux personnes à compétences
techniques dans le domaine de l'informatique, communément appelées
}\textbf{geeks} ou \textbf{\emph{nerds}}. Toutefois les novices peuvent
y lire des choses intéressantes pour eux. Ils sont donc invités dans ce
cas à faire des recherches et à avoir beaucoup de patience dans
celles-ci.

On a donc OpenPGP et ses implémentations logicielles, PGP et GPG, des
outils pour protéger son courriels des regards indiscrèts. Mais est-ce
le meilleur moyen ?

\section{Bitmessage}\label{bitmessage}

Il existe par exemple Bitmessage
\footnote{\url{http://www.bortzmeyer.org/bitmessage.html}}, qui est
pour moi l'anti-exemple parfait, puisqu'il ne respecte pas le
paradoxe de la moquette\footnote{\url{http://www.22decembre.eu/2015/02/23/carpet-paradox-fr/}} : les adresses ressemblent furieusement à des chaînes de
caractères aléatoires, il n'est donc pas aisé de donner son adresse à son interlocuteur.

Même donner son adresse sur papier présente le risque que votre interlocuteur se trompe en la recopiant.\\Il n'est donc rien de plus
facile que de se gourer et d'envoyer son message à quelqu'un d'autre.\\Je pense que le moyen le plus efficace est encore de copier
son adresse sur un site web (et encore faut-il être prudent !).

La base du fonctionnement du réseau Bitmessage lui-même est compréhensible. Mais dès qu'on essaye de comprendre davantage
(essentiel, lorsqu'il s'agit de protocoles de sécurité et de chiffrement), on se met à se gratter la tête !

\subsection{Un petit truc étrange, une réflexion, comme ça\ldots{}}\label{un-petit-truc-uxe9trange-une-ruxe9flexion-comme-uxe7a}

Il y a également un truc qui me chiffonne, outre l'aspect obscur, non-écologique, non-efficient au plan énergétique : le protocole
Bitmessage est conçu pour noyer les communications chiffrées dans le flot des données P2P bit-torrent.\\Ainsi, soit disant, la NSA (qui est
typiquement l'organisme auquel les concepteurs du protocole tentent d'échapper) ne verrait pas ce flot de données.

Mais justement, la NSA a maintenant connaissance d'un protocole de communications sécurisé utilisé par des hackers de haut-niveau !\\Enfin
du moins, elle doit être au courant : elle lit des courriels, visite les sites web et les forums\ldots{}

On \emph{doit} assumer qu'elle est au courant ! Donc quelqu'un a conçu
un protocole de communications dans le but d'échapper à la NSA, et ce
protocole envoie en permanence \emph{tous} les messages à \emph{tout} le
réseau.

Quoi de plus facile pour la NSA (et pour tous les autres maintenant) de
se créer une ou plusieurs adresses Bitmessage, moissoner le flux
indistinctement, et tenter de casser le code de chiffrement,
puisqu'apparemment, c'est ce qu'elle fait déjà \footnote{\url{http://www.nytimes.com/2013/09/06/us/nsa-foils-much-internet-encryption.html} - lien en anglais} !

Le code de Bitmessage repose, en partie sur SSL/TLS, une technologie commune et répandue, sur laquelle tout le monde travaille. Pas le plus
sûr comme je l'indique plus bas.

\subsection{GPG ou Bitmessage ?}\label{gpg-ou-bitmessage}

\begin{notice}
Je précise que ceci est mon opinion, mon utilisation de mes outils informatiques. Ce n'est pas un commandement sacré à respecter au pied de
la lettre. Il peut y avoir débat. Si vous voulez troller, libre à vous, mais ce sera sans moi !
\end{notice}

Je préfère donc utiliser GPG plutôt que Bitmessage. Le courriel est
légitime, je n'ai pas de raison de le cacher - même si je comprends les
arguments de ceux qui le veulent, entre autre cacher \emph{à qui on écrit}.

Utiliser Bitmessage me marque automatiquement comme un \emph{hacker} de
haut vol, ce que je ne suis pas. Tout juste puis-je et souhaite
prétendre au status de \emph{petit} hacker ou de \emph{padawan}\ldots{}

En utilisant Bitmessage, j'encourage des gens à surveiller mon courriel
et à tenter de le lire.

Au contraire, en écrivant des courriels signés et/ou chiffrés, j'assume
ma correspondance, et en même temps je la protège, ce qui est on ne peut
plus légitime. Et comme c'est toujours du courriel, ça incite mes
contacts à utiliser GPG\ldots{}

Et si quelqu'un s'amuse à décrypter mon courriel, il découvrira alors
probablement mes échanges avec mon amie lesbienne ou mes propositions de
projets dans le cadre de mon travail. Des choses hautement inutiles pour
NSA \& co !

Point bonus : à moi, ça me m'a demandé aucun effort, alors que mon amie la NSA a gâché X heures de temps de calcul dessus !

\begin{quoting}
Et faire chier les gens qu'on n'aime pas dès le matin, c'est vraiment gratifiant !
\end{quoting}

\emph{Extrait de l'interview de Coreight par Cyrille Borne}
\footnote{\url{http://cyrille-borne.com/article47/du-berger-a-la-bergere-l-interview-de-coreight}}

\section{Les autres systèmes de chiffrement classiques}\label{les-autres-systuxe8mes-de-chiffrement-classiques}

On peut chiffrer et authentifier son courrier avec SSL/TLS, par le biais
du format S/Mime. Certaines entités fournissent des certificats personnels gratuitement.

C'est le cas de DanID qui fournit le système d'authentification NemID, permettant d'accéder aux sites bancaires et gouvernementaux danois. Les
certificats fournis par NemID sont valides pour signer et chiffrer du courrier, et s'authentifier sur certains sites web, notamment
DBA, le Bon Coin danois, mais pas pour sécuriser son site web - hélas.

En apparence, ces certificats sont donc une bonne chose pour sécuriser son courrier (les certificats de NemID sont un premier pas en avant vers
l'identité virtuelle des citoyens après tout). Mais SSL et TLS sont assez sujets à critiques dernièrement, et la NSA (et d'autres agences de
renseignement certainement) a déjà cassé nombre de ses algorithmes apparemment.

J'ai tendance à me mefier de plus en plus de TLS, en plus de son manque
d'ergonomie, parce qu'apparemment c'est aussi solide qu'une passoire en
plastique et tout aussi troué.

\begin{notice}
	Néamoins, je recommande toujours l'utilisation de TLS dans la navigation web ! Un bouclier en plastique, c'est mieux que pas de bouclier du tout !
\end{notice}

SSL/TLS et GPG sont toutefois deux implémentations légerement différentes du même principe : chiffrement asymétrique, avec une clé
publique et une clé privée. Dans les faits, ce n'est donc pas vraiment plus facile, et certainement pas plus sûr d'utiliser SSL/TLS.

Pour le courriel, je pense qu'il vaut mieux utiliser un outil conçu pour ça, à savoir GPG.

\section{Retour à GPG}\label{retour-uxe0-gpg}

GPG lui s'efforce de respecter le paradoxe de la moquette : il s'appuie sur des protocoles connus, assez bien maîtrisés et facile à prendre en
main (le courriel).\\Les clés peuvent être identifiées par l'adresse
courriel et/ou l'empreinte. Ceci permet de les trouver facilement et rassure les utilisateurs.

Et il n'était pas cassé par la NSA au dernières nouvelles.