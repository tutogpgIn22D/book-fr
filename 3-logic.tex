\chapter{Un peu de théorie et de logique}

\section{OpenPGP, comment ça marche ?}\label{openpgp-comment-uxe7a-marche}

OpenPGP repose sur le principe de la cryptographie - ou chiffrement en français - asymétrique.

Quand on parle de clés GPG, on parle en fait de couples de clés : une clé publique et une clé privée.

\section{Pourquoi se compliquer la vie avec ça ?}\label{pourquoi-se-compliquer-la-vie-avec-uxe7a}

Lorsque vous signez un message, vous voulez garantir que celui-ci vient
bien de vous et qu'il n'a pas été altéré.\\C'est le même principe que
les sceaux au Moyen-Âge : le cachet de cire intact garantissait
l'authenticité et l'inviolabilité du message.

Vous devez donc signer votre message avec quelque chose que vous seul
avez en votre possession : votre clé \textbf{\emph{privée}} !

Et comment vos interlocuteurs vérifient-ils que ce message est
authentique ?\\Avec un élément qui est connu de tous, qui est public :
la vérification des signatures se fait avec votre clé
\textbf{\emph{publique}} !

Inversement, quand on chiffre un message à votre attention, on le rend
illisible pour toute personne n'en ayant pas la clé. Vous seul devez
pouvoir le lire avec quelque chose que vous seul avez en votre
possession : votre clé \textbf{\emph{privée}} !

Mais comme n'importe qui doit pouvoir vous envoyer des messages
chiffrés, l'opération de chiffrement doit se faire avec une information
connue de tous : votre clé \textbf{\emph{publique}} !

\section{Attention !}\label{attention}

Faîtes attention ! Un message signé \textbf{peut être lu par n'importe
qui sur le net}.

La signature garantit que vous êtes bien l'émetteur du message, et que
celui-ci n'a pas été altéré. Un message signé a donc plus de valeur
juridique qu'un message non signé.

\section{Récapitulatif}\label{ruxe9capitulatif}

On signe ses messages avec sa clé \emph{privée}. On vérifie
l'authenticité des messages d'autres personnes avec leur clé
\emph{publique}.

On chiffre les messages à destination d'autres personnes avec leur clé
\emph{publique}, qu'ils déchiffreront avec leur clé \emph{privée}.

Observez que c'est presque toujours le même côté de la liaison qui
utilise le même coté de la clé: vous utilisez presque toujours votre clé
privée, et vos correspondants n'utilisent que votre clé publique.

Votre clé privée doit donc être jalousement gardée et protégée !

\begin{figure}[h]
\centering
\includegraphics[width=0.5\linewidth]{./images/gollum.jpg}
\end{figure}

C'est elle qui vous permet de contrôler votre coté de la liaison, de
prouver votre identité et de lire votre courrier !