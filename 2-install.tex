\chapter{Installation des outils de base}

\section{Les logiciels nécessaires}\label{les-logiciels-nuxe9cessaires}

Pour pouvoir utiliser GPG, il vous faut \ldots{}

\begin{enumerate}
\def\labelenumi{\arabic{enumi}.}
\itemsep1pt\parskip0pt\parsep0pt
\item
  GPG
\item
  Un gestionnaire de clés tel que :

  \begin{itemize}
  \itemsep1pt\parskip0pt\parsep0pt
  \item
    Kgpg/Kleopatra
  \item
    Enigmail
  \end{itemize}
\item
  Un client courriel tel que :

  \begin{itemize}
  \itemsep1pt\parskip0pt\parsep0pt
  \item
    kmail
  \item
    Thunderbird
  \item
    Évolution
  \end{itemize}
\end{enumerate}

Très souvent, les gestionnaires de clés sont des logiciels additifs aux
clients courriels, des extensions. Ils ont pour la plupart les mêmes
options de base.\\Le choix de tel ou tel environnement s'avère donc
trivial, une affaire de goût et d'esthétique plutôt que de
fonctionnalités.

Kgpg est un élément de la suite KDE, qui s'utilise donc surtout avec
Kmail et Kontact. De même Enigmail fonctionne avec Thunderbird.

\begin{notice}
	Il est à noter que les tablettes et téléphones Samsung avec
Android ont maintenant tous les outils GPG nécessaires. Le problème
c'est que vous ne pouvez vraiment leur faire confiance.\\
\\
C'est néanmoins un bon moyen pour faire votre initiation à GPG.
Vous voudrez certainement recréer des clés après ça sur un vrai
ordinateur.
\end{notice}

Au passage, vous pouvez envisager, si vous êtes vraiment motivé,
d'installer votre propre système de courriel.\\Ça devient accessible au
non-informaticien, grâce à YunoHost \footnote{\url{https://yunohost.org/}} par
exemple.\\Ou dans une autre philosophie, OpenBSD, très simple
d'installation\footnote{\url{http://blog.chown.me/pourquoi-j-adore-openbsd.html}}, mais un poil difficile à prendre en main. Toutefois, pas plus difficile, de base, qu'une Debian.

\section{Recommandations}\label{recommandations}

\begin{notice}
La plupart de ces recommandations sont d'ordre général. Il s'agit de ce
qu'on pourrait appeler ``des pratiques de base saines pour votre
ordinateur''.
\end{notice}

\subsection{Sécurité du courriel}
En gros, il faut éviter le webmail, qui est une très mauvaise chose -
déjà de base - puisque vous accédez à votre courriel via le web, donc on
ne maîtrise pas la sécurité. Sans même parler d'utiliser GPG là dedans !

Le chiffrement et les signatures de vos courriels ne vous protégeront
nullement contre les virus et autres saloperies qui traînent ! Ils vous
garantiront que vos courriels sont bien authentiques et/ou n'ont pas été
lus par une autre personne que le destinataire légitime.

\subsection{Sécurité des logiciels}
Il faut également éviter de récupérer des logiciels sur des sites tiers
comme 01net et telecharger.com.\\Il est plutôt recommandé d'aller voir
le site web du concepteur du logiciel - tout en restant prudent\footnote{\url{http://cyrille-borne.com/article398/le-site-de-confiance-c-est-termine}}, ou un site officiel, comme le site web d'Apple pour le cas d'un logiciel pour Mac
par exemple.

Lorsque vous pouvez utiliser un logiciel libre ou open-source plutôt
qu'un logiciel propriétaire, faîtes le. C'est d'autant plus important
dans le cadre de protocoles critiques (sécurité\ldots{}).\\En effet,
avec un code en libre accès, n'importe qui peut vérifier la qualité du
code, garantir l'absence de portes dérobées, de pratiques douteuses.\\Un
code source ouvert (\emph{opensource}) signifie donc que vous pouvez lui
faire confiance pour \emph{vous protéger, vous et votre vie privée} ! Et
si vraiment vous êtes paranoïaque (c'est votre droit) alors vous pouvez
prendre des cours de code, puis faire vous même cette vérification !

Lors de l'installation d'un logiciel, ne faîtes pas «entrée» à tout va.
Regardez les options.\\Il est fréquent qu'un installateur vous propose
une barre d'outils ou autre, que vous n'avez pas demandé, et dont vous
n'avez pas besoin.\\Ces micro-ajouts permanents sont source de
ralentissements multiples, et contiennent parfois des logiciels espions.

Certains auteurs de logiciels signent leurs binaires (le fichier à
télécharger) avec leur clé gpg ou indiquent les sommes de contrôle MD5
ou SHA. Je ne vous ai pas encore expliqué comment vérifier les
signatures, mais si vous savez le faire, faîtes le !

\subsection{autres liens et documentations utiles}

SPF a écrit un guide pour la navigation internet plus sûre\footnote{\url{http://sanspseudofix.fr/kit-de-base-du-surf-tranquille-2/}}.
Genma (un fervent partisan de l'utilisation des outils de chiffrement) a son guide d'hygiène numérique\footnote{\url{http://genma.free.fr/?Petit-guide-d-hygiene-numerique}} \footnote{\url{https://github.com/genma/Conference_Guide_d_hygiene_numerique/blob/master/Genma_Petit_Guide_d_hygiene_numerique.pdf?raw=true}}.

\section{GPG}\label{gpg}

\subsection{Linux and co}\label{linux-and-co}

Gpg est dans les standards des distributions Linux. Si vous ne l'avez
pas, c'est que votre distro est tellement particulière que je n'en
connais pas le mode d'installation ou de gestion des paquets (
\emph{Slitaz} ?).

Sous Debian \& co, si c'est pas déjà installé - ce qui serait bizarre
puisque Debian utilise GPG pour signer et vérifier l'intégrité et
l'authenticité des paquetages logiciels (!) , ça donne ça :

\begin{lstlisting}
apt-get install gnupg gnupg2
\end{lstlisting}

J'indique les deux paquets. La version 2 est celle recommandée
aujourd'hui.

De toute façon, il est dans les dépendances de tous les gestionnaires de
clés qu'on verra plus loin.

Les utilisateurs d'autres distributions ou d'outils graphiques tels que
Synaptic, Apper ou Muom feront simplement une recherche sur
\textbf{gnupg}. Il y a de fortes chances que votre gestionnaire de
paquets vous dise qu'il est déjà installé.

Il est aussi dans la base des Unix tel qu'OpenBSD.

\subsection{Windows}\label{windows}

Bon, là on s'attaque à un gros morceau !

Il vous faut en fait Gpg4win \footnote{\url{http://www.gpg4win.org/download.html} - prenez la première version en haut, sauf si vous savez ce que vous
faîtes}, qui contient d'ailleurs tout le nécessaire\footnote{\url{http://www.gpg4win.org/about.html}}: gestionnaire de clés, client
courriel\ldots{}

Une fois l'installateur téléchargé, lancez le. Il vous proposera d'installer d'autres logiciels en plus de GPG.

\begin{itemize}
\itemsep1pt\parskip0pt\parsep0pt
\item
  GPA est un gestionnaire de clés
\item
  Kleopatra est un autre gestionnaire de clés
\end{itemize}

Vous avez besoin d'un des deux. Kleopatra est le plus décrit sur le web,
c'est donc celui que je conseille (et je l'ai sous la main, si vous me
demandez de l'aide, je pourrais plus facilement vous aider).

\begin{itemize}
\itemsep1pt\parskip0pt\parsep0pt
\item
  GpgOL, plugin pour Outlook
\item
  GpgEX, plugin pour l'explorateur de fichier de Windows.
\end{itemize}

Installez les si vous avez besoin.

\begin{itemize}
\itemsep1pt\parskip0pt\parsep0pt
\item
  Claws-Mail, logiciel de courrier léger
\end{itemize}

Les windowsiens, on vous facilite la vie décidément !

\subsection{MacOS}\label{macos}

Je n'ai pas de Mac sous la main. Mais Thunderbird est disponible sous
Mac et vous pouvez donc l'utiliser.\\Apparemment
le client courriel natif de Mac supporte également le chiffrement\footnote{\url{http://www.gbronner.net/mail/GPGMacOSX.html}}.

Vous avez besoin, des outils de chiffrements Gpg pour Mac. Téléchargez donc la suite
gpg\footnote{\url{https://gpgtools.org/}}.

Vous pouvez (bon en fait vous \emph{devriez}) vérifier l'intégrité du
fichier dmg en allant dans votre dossier de téléchargement (je suppose
ici qu'il s'appelle \emph{downloads} ) dans votre terminal :

\begin{lstlisting}
cd downloads
openssl sha1 GPG_Suite
\end{lstlisting}

En tapant le nom du fichier, vous pouvez après faire auto-complétion :
utilisez la touche de tabulation, le terminal complétera le nom du
fichier.

La commande ssl va vous indiquer une suite de caractères qui doivent
correspondre à celle indiquée sur le site de gpgtools en dessus du
bouton de téléchargement.

Reste plus qu'à installer. Ouvrez le fichier dmg et cochez ou décochez
les options qui vont bien.\\Il vous faut \emph{MacGPG2},
\emph{GPGPreferences}, \emph{GPG Keychain Access}.

Si vous utilisez le logiciel natif \textbf{Mail}, il vous faut aussi
\textbf{GPG for Mail}, mais si vous utilisez Thunderbird, il vous faut
par contre \textbf{GPG Services}.

\section{Un client couriel}\label{un-client-couriel}

\begin{notice}
Je ne décrirais pas la configuration de la messagerie, des adresses
courriels.\\Si vous êtes venus jusqu'ici, ce dont je vous félicite,
c'est que vous êtes motivés pour apprendre/chercher la solution par
vous-même et/ou que vous savez déjà configurer une adresse courriel.

Toutefois, on peut toujours me contacter\footnote{via l'adresse du tutoriel, ou aller sur \url{http://http://www.22decembre.eu/fr/contact.html}} pour demander de
l'aide. Les tutoriels pour la configuration de la messagerie de
Thunderbird (aisement transposable aux autres logiciels de courrier) sont légions sur le web.\footnote{\url{http://www.astucesinternet.com/modules/news/article.php?storyid=180} par exemple}.
\end{notice}

Bon, vous avez le logiciel de base installé, mais rien d'autre pour
l'instant à priori. Prenez le client courriel de votre choix.

\subsection{Thunderbird}\label{thunderbird}

Thunderbird est disponible en téléchargement et installation pour toutes
les plates-formes majeures\footnote{lien vers la page de téléchargement universelle: \url{https://www.mozilla.org/en-US/thunderbird/all.html}.
	Vous l'avez ici en français, et pour votre système: \url{https://www.mozilla.org/fr/thunderbird/}}.

Vous pouvez donc l'installer avec apt\footnote{Il est à noter que le logiciel est renommé Icedove sous Debian suite à la
	controverse entre Debian et Mozilla: \url{http://fr.wikipedia.org/wiki/Renommage_des_applications_de_Mozilla_par_Debian}} :

\begin{lstlisting}
apt-get install icedove
\end{lstlisting}

\subsection{Claws-Mail}\label{claws-mail}

Si vous êtes sous Windows, vous avez le client \textbf{Claws-Mail} dans
l'installateur de \textbf{Gpg4win}

\subsection{Kmail et Évolution}\label{kmail-et-uxe9volution}

Kmail est disponible comme partie de la distribution KDE, Évolution
comme partie de Gnome, donc si vous êtes sous GNU/Linux, vous devriez
utiliser votre gestionnaire de paquet favori.

\begin{lstlisting}
apt-get install kmail

apt-get install evolution
\end{lstlisting}

Même remarque qu'auparavant pour ce qui est des installateurs graphiques
: Synaptic et consorts installeront toutes les dépendances, y compris
\textbf{gnupg} si ce n'est pas encore le cas.

\subsection{Les autres}\label{les-autres}

Il existe une version Windows de KDE\footnote{\url{https://windows.kde.org/}},
mais je n'ai jamais pris le temps de l'essayer.

Sylpheed, client courriel disponible sous distributions Linux, Windows,
Mac et d'autres Unix.

\section{Un gestionnaire de clés}\label{un-gestionnaire-de-cluxe9s}

\begin{notice}
Rappel : Vous n'avez besoin que d'un seul de ces logiciels. Et très
souvent le choix de tel ou tel logiciel dépend de votre environnement !
\end{notice}

\subsection{GPG Keychain sous Mac OS}\label{gpg-keychain-sous-mac-os}

Le gestionnaire de clés s'appelle GPG Keychain sous Mac Os et vous
l'avez normalement déjà installé lorsque vous avez installé GPG pour
Mac.

\subsection{Thunderbird : Enigmail}\label{thunderbird-enigmail}

Si vous utilisez Thunderbird, vous avez besoin d'Énigmail, qui est en
fait une extension, un plugin du logiciel utilisé par Thunderbird pour
gérer son interaction avec GPG.

Les utilisateurs de Thunderbird peuvent suivre le tutoriel du Hollandais Vollant
\footnote{\url{http://lehollandaisvolant.net/tuto/gpg/\#i2}}, qui est
complet et dont je m'inspire beaucoup. Toutefois je le trouve indigeste
car très long.

\subsubsection{Première solution : le site web}\label{premiuxe8re-solution-le-site-web}

L'installation est ici la même que pour l'installation d'un module
Firefox : xpi.\footnote{Il vous faut télécharger le plugin sur
\url{https://www.enigmail.net/download/} et bien sûr
prendre la version correspondant à votre OS.} 
Au passage, signalons comme l'indique bien
la page de traduction\footnote{\url{http://beta.babelzilla.org/projects/p/Enigmail/}} qu'Énigmail est traduit dans de nombreuses langues, y
compris français (90\%), mais pas danois, hélas.

L'extension xpi s'installe comme suit : il faut démarrer Thunderbird,
sélectionner «Outils» dans la barre des menus en haut, puis «plugins»,
«extensions» ou «modules».\\Ou alors (suivant votre version de
Thunderbird) cliquer sur le gros bouton en haut à droite, et
sélectionner «plugins», «extensions» ou «modules».

Ici, vous pouvez indiquer à Thunderbird que vous souhaiter installer une
extension en cliquant en bas à gauche sur «installer\ldots{}».
Thunderbird vous demandera dans quel dossier de votre ordinateur vous
avez téléchargé le fichier xpi d'Énigmail.

Une fois cette opération réalisée, il faut redémarrer Thunderbird.

\subsubsection{Deuxième solution : télécharger via Thunderbird lui-même}\label{deuxiuxe8me-solution-tuxe9luxe9charger-via-thunderbird-lui-muxeame}

Vous pouvez demander à Thunderbird/Icedove de vous télécharger et
installer l'extension par lui-même.

Allez dans la fenêtre des modules, comme indiqué précédemment et faîtes
une recherche sur Enigmail. Normalement vous devriez l'avoir en tête de
liste avec un bouton d'installation.

\subsubsection{Bonus Debian : installation via Apt}\label{bonus-debian-installation-via-apt}

Si vous utilisez Icedove (sic !) sous Debian, sachez qu'Énigmail est
dans les dépots Debian et que vous pouvez donc l'installer via apt.
C'est une bonne solution si vous partagez votre ordinateur avec d'autres
utilisateurs.

Par contre, je pense que si on installe Énigmail par ce biais, il faut
alors éviter à tout prix de l'installer après par un autre moyen si vous
voulez mettre à jour.

Si vous souhaitez une autre version que celle des dépots, il vous faut
la désinstaller au préalable.

\begin{warning}
Une seule version de ce logiciel par machine ! C'est, je
pense, une mesure de sécurité.
\end{warning}

\begin{lstlisting}
apt-get install enigmail
\end{lstlisting}

\begin{notice}NB : Il n'y a pas de raison, si Debian a inclus Énigmail dans ses
dépots que d'autres distributions GNU/Linux ne l'ai pas
fait\ldots{}\\Essayez donc de chercher Énigmail dans votre gestionnaire
de paquets.
\end{notice}

\subsection{Kgpg/Kleopatra}\label{kgpgkleopatra}

Kgpg et Kleopatra sont deux logiciels de gestion des clés et certificats
de chiffrement sous le bureau KDE. Pour ma part, je préfère me servir de
Kgpg, mais il est quand même très utile d'avoir les deux installés.

Il y a de fortes chances, si vous êtes un fan de KDE comme moi que ces
logiciels soient déjà installés. Mais sinon, comme précédemment, et
suivant votre distribution :

\begin{lstlisting}
apt-get install kgpg kleopatra
\end{lstlisting}

Rappelez-vous que Kleopatra est aussi distribué dans l'installateur
Gpg4win sous Windows.

\subsection{Les autres}\label{les-autres-1}

Il y a Seahorse sous Gnome, mais je ne connais pas du tout. Mais je
doute fortement qu'il soit très différent de Kgpg. Il est disponible
sous Debian :

\begin{lstlisting}
apt-get install seahorse
\end{lstlisting}
