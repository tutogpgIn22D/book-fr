\chapter{Signer son courriel}

Vous avez suivi jusqu'ici et avez donc une clé gpg. Excellent !

Vous pouvez aussi demander de l'aide par courriel à l'adresse du
tutoriel : \emph{Tuto-gpg @ 22decembre.eu}.

On va maintenant faire en sorte que vous puissiez utiliser votre clé
dans votre courriel.

\section{Mais comment faire pour que vos divers contacts puissent utiliser votre clé ?}\label{mais-comment-faire-pour-que-vos-divers-contacts-puissent-utiliser-votre-cluxe9}

\subsection{Config' du client courriel}\label{config-du-client-courriel}

Déjà, il faut leur signaler l'existence de gpg.\\Un petit texte en bas
de votre courriel suffira. Ce texte est généralement appelé
\emph{signature}. Il faut bien le différencier de la signature par GPG
de vos messages.

Voici ma signature française par exemple :

\begin{quote}
Ce fichier \textbf{\emph{signature.asc}} ? C'est une signature GPG.\\Si
vous voulez savoir pourquoi j'utilise GPG et pourquoi vous le devriez
aussi, vous pouvez lire mon article
:\\http://www.22decembre.eu/2015/03/21/introduction-fr/
\end{quote}

Ce texte est à renseigner dans les options de votre client courriel.
C'est également là que vous indiquez vos options pour vos signatures
GPG.

\subsubsection{Kmail}\label{kmail}

Dans Kmail, c'est \textbf{\emph{Configuration \textgreater{} Configurer
Kmail \textgreater{} Identités}}.

Là, vous trouverez les options de chiffrement, où vous indiquez quelle
clé \textbf{\emph{privée}} vous utilisez pour signer vos messages. Vous
indiquez aussi ici quelle clé \textbf{\emph{publique}} est à utiliser
pour chiffrer les messages à destination de vous-même (excellent moyen
pour partager une info, un mot de passe entre plusieurs ordinateurs).

Sur la question du format, il
\href{http://blog.chown.me/choisir-pgp-mime-ou-pgp-inline.html}{vaut
mieux} utiliser le chiffrement \emph{OpenPGP/Mime}, plutôt que
\emph{inline}.

Il est également important d'indiquer vos préférences quand à la
rédaction des messages dans \textbf{\emph{Configuration \textgreater{}
Configurer Kmail \textgreater{} Sécurité \textgreater{} Rédaction}}.
Moi, j'ai presque tout coché (signer par défaut, chiffrer quand c'est
possible\ldots{}) sauf \emph{Toujours afficher les clés de chiffrement}.

\subsubsection{Thunderbird}\label{thunderbird}

Dans \textbf{\emph{Outils \textgreater{} Paramètres des comptes}},
sélectionnez le menu \emph{Sécurité OpenPGP} sous l'adresse qui vous
plaît. Cochez l'option \emph{Activer le support OpenPGP (Enigmail) pour
cette identité} ainsi que l'option \emph{Utiliser l'adresse électronique
de cette identité} pour identifier la clef OpenPGP.

Vous pouvez ensuite choisir vos options par défaut : chiffrer, signer
tous vos courriels ou non, et si vous voulez utiliser le format
\emph{PGP/Mime}, ce que je recommande (même remarque et même
\href{http://blog.chown.me/choisir-pgp-mime-ou-pgp-inline.html}{lien}
qu'au dessus).

Vous avez aussi des options à choisir dans \textbf{\emph{Enigmail
\textgreater{} Préférences}}.

\subsubsection{Évolution}\label{uxe9volution}

\emph{Édition \textgreater{} Préférences}

Dans la fenêtre dans l'onglet \emph{Comptes de messagerie}, puis le
compte en question et cliquer sur \emph{modifier}.\\Dans l'Editeur de
comptes qui s'ouvre, aller dans l'onglet \emph{Sécurité}.\\Dans le champ
\emph{ID de la clé PGP/GPG} : entrer l'identifiant en 8 caractères tel
que récupéré dans votre gestionnaire de clés.

Pensez aussi à mettre vos options par défaut.

Lors de la rédaction d'un nouveau message, dans le menu \emph{Options},
cliquer sur \emph{Chiffrer} ou \emph{Signer}.

\section{Faut-il signer et chiffrer tout son courriel ?}\label{faut-il-signer-et-chiffrer-tout-son-courriel}

La question est plus ou moins de nature philosophique et constitue un
choix personnel.\\De toute façon le logiciel de courriel a très
certainement de gros boutons qui n'attendent que d'être utilisés pour
réaliser ces opérations.

Pour ma part, j'aime bien cette remarque de Philip Zimmermann :

\begin{quote}
Que se passerait-il si tout le monde estimait que les citoyens honnêtes
devraient utiliser des cartes postales pour leur courrier? Si un
non-conformiste s'avisait alors d'imposer le respect de son intimité en
utilisant une enveloppe, cela attirerait la suspicion. Peut-être que les
autorités ouvriraient son courrier pour voir ce que cette personne
cache.\\Heureusement, nous ne vivons pas dans ce genre de société car
chacun protège la plupart de son courrier avec des enveloppes.\\Aussi
personne n'attire la suspicion en protégeant son intimité avec une
enveloppe. La sécurité vient du nombre.\\De la même manière, ce serait
excellent si tout le monde utilisait la cryptographie de manière
systématique pour tous ses e-mails, qu'ils soient innocents ou non, de
telle sorte que personne n'attirerait la suspicion en protégeant
l'intimité de ses e-mails par la cryptographie.\\Pensez à le faire comme
une forme de solidarité.
\end{quote}

Dans tous les cas, vous pouvez signer tous vos courriers sans que cela
pose de problème pour vos correspondants (normalement - il est possible
qu'ils en aient s'ils utilisent un mauvais client de messagerie. Mais
c'est extrêmement rare !).

\textbf{Un courriel signé est authentique. Votre correspondant aura la
certitude qu'il vient bien de vous et qu'il n'a pas été changé au cours
de son acheminement. En revanche, il n'est pas encore possible de
garantir que personne d'autre ne l'ait lû !}

En revanche, vous ne pouvez chiffrer vos courriers que si vos
correspondants utilisent eux aussi GPG ; puisque, rappelez-vous : vous
avez besoin de leur clé publique pour chiffrer des messages à leur
attention.

\section{Exercice}\label{exercice}

Bon, aujourd'hui, je vais vous demander de \emph{signer} un courriel à
mon intention. Tout simplement !

C'est pour cette raison que je vous ai demandé de m'envoyer la clé que
vous avez généré lors de la lecture du précédent article : j'en ai
besoin pour vérifier votre signature, donc pour vérifier que vous avez
bien compris cette partie du tutoriel.

Donc, revenons à l'exercice. Pour envoyer un courriel signé, ouvrez
votre logiciel de courriel, et écrivez un message à \emph{Tuto-gpg @
22decembre.eu}.

Vous pouvez écrire ce que vous souhaitez, y compris une critique du
tutoriel. Mais dans ce cas, j'aimerais qu'elle soit constructive,
qu'elle me permette de l'améliorer.

Avant de l'envoyer, sélectionnez \emph{Signer} dans les options ou
utiliser le bouton adéquat.\\Si (comme moi) vous avez demandé à votre
logiciel de courriel de signer tous vos messages, vous n'avez rien à
faire en fait ! Sauf appuyer sur le bouton \emph{Envoyer}\ldots{}

Quand vous aurez envoyé votre courriel, passez donc à
\href{\{filename\}6-crypted-mail-fr.md}{l'article suivant}.

Au plaisir de vous lire.
